As integer-valued invariants, Betti numbers pose several difficulties to vectorization. For example, the rank function typically defined on matrices $M$ is given as: 

Much is known about the spectrum of Laplacian matrices. From discerning the beat of drums to understanding the chemical makeup of Benzene.... etc, the normalize graph Laplacian has found many interpretations of its spectrum. 

Typically, one often needs substantial and precise domain-specific knowledge about the phenonum in order to write it down as, e.g. a partial differential equation (PDE) and to interpret its corresponding Laplace operator(s). For example, one may need to know the underyling system producing the observed data follow some prescribed laws of motion, or conversation laws of energy, etc.
However, information about such laws is either absent, obfuscated, or non-existent in many modern data science applications. This is invariably the case for many human-generated data sets: movie ratings, newsfeeds, social media updates, etc.
 
 implicitly regarded a graph as a discrete analogue of a
Riemannian manifold and cohomology as a discrete analogue of PDEs: standard
partial differential operators on Riemannian manifolds — gradient, divergence, curl,
Jacobian, Hessian, scalar and vector Laplace operators, Hodge Laplacians — all have
natural counterparts on graphs
 

For example, it is known that one of them completely characterizes the other 2 sets...~\eqref{} 
% Include 2.5 from Laplacian Simplicial complex paper 

%This is perhaps best illustrated.... The excessive freedom associated with pure topological equivalence makes discrimination difficult. Contrary to a topologists intuition, we seek a relaxation whose sensitivity to geometry is adjustable. 
