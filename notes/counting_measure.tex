\newcommand\restr[2]{{% we make the whole thing an ordinary symbol
  \left.\kern-\nulldelimiterspace % automatically resize the bar with \right
  #1 % the function
  \vphantom{\big|} % pretend it's a little taller at normal size
  \right|_{#2} % this is the delimiter
  }}
  
 %For any $i,j \in \Delta_+$, define the \emph{multiplicity} $\mu_p^{i,j}$ of the pair $p = (i,j)$ by: 
%\begin{equation}
%	\mu_p^{i,j} = \beta_{p}^{i-1, j} - \beta_{p}^{i, j} + \beta_{p}^{i, j-1}  - \beta_{p}^{i-1, j-1}
%\end{equation}
%The multiplicity function is intimately linked with the persistence diagram. Indeed, the original definition of a persistence 

Persistence has been viewed from multiple, equivalent perspectives; we recall a few which are illuminating in this effort.
Among the first characterizations of the PBNs were given from an algebraic perspective---Carlsson et al.~\cite{zomorodian2004computing} observed that the persistent homology groups over a filtration may be viewed as the standard homology groups of a particular graded module $M$ over a polynomial ring. 
%More recently, Bauer studied persistence in a form a matching. 
In~\cite{cohen2005stability}, Cohen-Steiner et al. give a more discrete perspective by defining the persistence diagram in terms of a \emph{multiplicities}: given a tame function $f: K \to \mathbb{R}$, its homological critical values $\{ a_i \}_{i=1}^n$, and an interleaved sequence $\{ b_i \}_{i=0}^n$ satisfying $b_{i-1} < a_i < b_i$ for all $i$, the $p$-th persistence diagram $\mathrm{dgm}_p(f) \subset \bar{\mathbb{R}}^2$ of a filtration induced by $f$ is defined as: 
\begin{equation}
\mathrm{dgm}_p(K_\bullet) = \{ \, (a_i, a_j) :  \mu_p^{i,j} \neq 0 \, \} \; \cup \; \Delta	
\end{equation}
%the set of points $(a_i,a_j)$ drawn on the plane with non-zero multiplicity $\mu_p^{i,j}$: 
where $\Delta$ denotes the diagonal, counted with infinite multiplicity, and $\mu_p^{i,j}$ is the  \emph{multiplicity function}, defined as: 
\begin{equation}\label{eq:multiplicity}
	\mu_p^{i,j} = \left(\beta_p^{i,j\shortminus1} - \beta_p^{i,j} \right) - \left(\beta_p^{i\shortminus1,j\shortminus1} - \beta_p^{i\shortminus1,j} \right) \quad\quad \text{for } 0 < i < j \leq n+ 1
\end{equation}
For $M$ a decomposable persistence module over $\mathbb{R}$, Chazal~\cite{chazal2016structure} redefine the multiplicity function by interpreting $\mu_p^\ast$ as a \emph{persistence measure}:
\begin{equation}\label{eq:measure}
	\mu_p(R; M) = \mathrm{card}\left( \,
	%\mathrm{dgm}_p(M)\restriction_R 
	\restr{\mathrm{dgm}_p(M)}{R} \,
	\right) \quad \text{ for all rectangles } R \subset \mathbb{R}^2 
\end{equation}
The interpretation that the measure-theoretic perspective from~\eqref{eq:measure} provides is that $\mu_p$ is a kind of integer-valued measure defined over rectangles in the plane. Cerri et al.~\cite{cerri2013betti} incorporate this interpretation in their work studying the stability of PBNs in multidimensional persistence settings. Namely, they show that \emph{proper cornerpoints} in the persistence diagram are points $x = (i,j) \in \Delta_+$ satisfying:
%as limit points of the $\mu_p$, i.e. for every point $x = (i,j) \in \Delta_+$, define the number $\mu_p(x)$ as the minimum over all positive real numbers $\epsilon$ with $i + \epsilon < j - \epsilon$:
\begin{equation}\label{eq:pbn_cont}
	x = (i,j) \in \mathrm{dgm}_p(K_\bullet) \; \Longleftrightarrow \; \mu_p(x) > 0 \; \Longleftrightarrow \; \min_{\epsilon > 0} \left(\beta_p^{i \+ \epsilon, j \shortminus \epsilon} - \beta_p^{i \+ \epsilon, j \+ \epsilon} \right) - \left(\beta_p^{i \shortminus \epsilon, j \shortminus \epsilon} - \beta_p^{i \shortminus \epsilon, j \+ \epsilon} \right) > 0
\end{equation}
One of the primary contributions from~\cite{cerri2013betti} is a representation theorem expressing the persistent Betti number function as a sum of multiplicity functions. A consequence of this theorem is that any distance between persistence diagrams induces a distance between PBNs, a result which may be used to show that PBNs are stable functions on continuous filtering functions. 
That is, small changes in continuous scalar-valued filtering functions imply small changes in the corresponding persistent Betti numbers functions. 
This justifies the study of PBN functions in continuous, parameterized settings. 

\textbf{Computation:} Despite the stability of the PBNs in parameterized settings, a number of barriers prevent their application in real-world settings. For instance, the main stability theorem from~\cite{cohen2005stability} assumes that the underlying filtering function or persistence module is \emph{tame} and are defined over triangulable spaces. When the number of corner points is finite, the corresponding distance between PBNs reduces to the bottleneck distance from~\cite{cohen2005stability}. 
By assuming the filtering functions are just continuous (as in equation~\eqref{eq:pbn_cont}), instead of tame, we encounter two problems on the practical side: 1. the number of corner points is countably infinite, and 2. the matching distance between PBN functions $\beta_p(f), \, \beta_p(g)$ requires as input knowledge of the diagrams $\mathrm{dgm}_p(f), \, \mathrm{dgm}_p(g)$ themselves. 
% Moreover, achieving the infimum in ... requires computing the optimal bijection ... between D(f), D(g), which takes O(...). 
%one arrives at definitions of PBNs in the flavor of equation~\eqref{pbn_cont}. 




%Construct the persistence measure $\mu$
%we need to know that such a multiset exists and is unique. This is the
%content of Theorem 2.8, under the hypothesis that µV is finite and additive.



%let DgmpVq be a multiset in the half-plane such that (*) holds for all rectangles R.
